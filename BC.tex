% třída dokumentu je beamer
\documentclass{beamer}

% standardní balíčky pro češtinu a kódování UTF-8
\usepackage{ucs,czech}
\usepackage[utf8x]{inputenc}

% balíček pro použití tučného písma v prezentaci
\usepackage{beamerthemesplit}

% balíček umožňující vkládání obrázků .jpg
\usepackage{graphicx}

% balíček umožňující odkazy v prezentaci
\usepackage{hyperref}
\usepackage{verbatim}
% použité téma prezentace
\usetheme{Berkeley}

% nastavení hlavičky v prezentaci
\title{Tvorba distribuce OS Linux}
\subtitle{Building Linux distribution}
\author{\textbf{Martin Zajíc}}
\date{\today}

%%%%%%%%%%%% vlastní prezentace %%%%%%%%%%%%
\begin{document}

% 1. snímek obsahující titulní stránku
\frame{\titlepage
\noindent\scriptsize Vedoucí práce: Sysel Martin, doc. Ing. Ph.D. \\
Oponent práce: Dulík Tomáš, Ing.
}

% 2. snímek s obsahem prezentace
\section[Obsah]{}
\frame{
  % titulek snímku 
  \frametitle{Obsah prezentace}
  % vysázení obsahu
  \tableofcontents
}

% 3. snímek 
\section{Bakalářská práce}
% podtitul snímku
\subsection{Cíle}
% snímek má své interní jméno 'zacatek'
\frame[label=zacatek]
{
  \frametitle{Bakalářská práce}
  \framesubtitle{Cíle bakalářské práce}
Hlavním cílem této bakalářské práce bylo:
  % nečíslovaný seznam	
\begin{itemize}
 \item Vytvoření scriptů pro automatické sestavení distribuce operačního systému Linux podle návodu Linux From Scratch.
 \item Sepsání návodu na sestavení Linux From Scratch za pomoci vytvořených scriptů
\end{itemize}
 

}

% 4. snímek
\section{Úvod do problematiky}
\frame{
  \frametitle{Úvod do problematiky}
  \setbeamercovered{transparent}
        
  OS Linux:
  % nečíslovaný seznam se zvýrazněním a automatickým zobrazením
  \begin{itemize}[<+- | alert@+>]
    \item Distribuce
    \item Linux From Scratch
  \end{itemize}

  BASH:
  \begin{itemize}[<+- | alert@+>]
    \item Jazyk BASH
    \item Integrace s OS Linux
  \end{itemize}
}

% 5. snímek
\section{Praktická část}
\subsection{LFS by bash scripts}
\frame{
  \frametitle{LFS by bash scripts}
  \begin{itemize}
    \item Rozčlenění podle knihy LFS
    \item Až na několik přerušení plně automatické sestavení
    \item Snaha o~maximálně jednoduché a~srozumitelné scripty
    \item Možnost krokovaní téměř každého příkazu
    \item Možnost nainstalování grafického serveru
  \end{itemize}
%  \hyperlink{zacatek<5>}{\beamergotobutton{Návrat na úvod}}
}
\subsection{onlyunpackscript}
\begin{frame}[fragile]
  \frametitle{onlyunpackscript}
\begin{itemize}
 \item Cesta mezi automatickým sestavení a~plně ručním
 \item Odstraňuje opakující se část stahování a~rozbalování
 \item Každý balíček v~samostatné proceduře, kterou volá postupně cyklus
 \item Možnost obnovení při přerušení
\end{itemize}

\scriptsize\begin{verbatim}
  while [ $BP -le 85 ]
  do
    package_$BP
    unpackPackage
    waitUser
    save_break_point
    BP=`expr $BP + 1`
  done
\end{verbatim} 
%  \hyperlink{zacatek<5>}{\beamergotobutton{Návrat na úvod}}

\end{frame}

% 6. snímek
\section{Závěr}
\frame{
  \frametitle{Závěr}
\begin{itemize}
 \item Všechny body zadání se podařilo splnit.
 \item Praktické využití
\begin{itemize}
 \item Zjednodušení práce pro studenty a~ostatní zájemce
 \item Zlepšení znalostí OS Linux a~jazyka bash
\end{itemize}
 \item Celkový počet řádků kódu: 12107
\end{itemize}
}
% 8. snímek
\section{Dotazy}
\frame{
  \frametitle{Prostor pro vaše dotazy}
\Huge{Prostor pro vaše dotazy}
\begin{figure}
 \includegraphics[scale=0.7]{./tux_question.png}
\caption{}
\end{figure} 
}

% 8. snímek
\section{Děkuji za pozornost}
\frame{
  \frametitle{Děkuji za pozornost}
\Huge{Děkuji za pozornost}
}

\end{document}
