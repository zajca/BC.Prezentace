% třída dokumentu je beamer
\documentclass{beamer}

% standardní balíčky pro češtinu a kódování UTF-8
\usepackage{ucs,czech}
\usepackage[utf8x]{inputenc}

% balíček pro použití tučného písma v prezentaci
\usepackage{beamerthemesplit}

% balíček umožňující vkládání obrázků .jpg
\usepackage{graphicx}

% balíček umožňující odkazy v prezentaci
\usepackage{hyperref}

% použité téma prezentace
\usetheme{Berkeley}

% nastavení hlavičky v prezentaci
\title{Tvorba distribuce OS Linux}
\subtitle{Building Linux distribution}
\author{Martin Zajíc}
\date{\today}

%%%%%%%%%%%% vlastní prezentace %%%%%%%%%%%%
\begin{document}

% 1. snímek obsahující titulní stránku
\frame{\titlepage}

% 2. snímek s obsahem prezentace
%\section[Obsah]{}
%\frame{
%  % titulek snímku 
%  \frametitle{Obsah prezentace}
%  % vysázení obsahu
%  \tableofcontents
%}

% 3. snímek 
\section{Bakalářská práce}
% podtitul snímku
\subsection{Zadání}
% snímek má své interní jméno 'zacatek'
\frame[label=zacatek]
{
  \frametitle{Bakalářská práce}
  \framesubtitle{Oficiální zadání bakalářské práce}

  % nečíslovaný seznam	
  \begin{itemize}
    \item Sestavte distribuci OS Linux, použijte návod Linux From Scratch.
    \item Napište scripty ve scriptovacím jazyce BASH určené pro automatické sestavení Linux From Scratch.
    \item Zdokumentuje postup sestavení Linux From Scratch pomocí BASH scriptů zopakovatelný pro studenty bakalářského studia.
    \item Popište základní odlišnosti Linuxových distribucí a možnosti tvorby Linuxové distribuce.
    \item Věnujte kapitolu problematice zabezpečení linuxového operačního systému.
    \item Vypracujte postup pro kompilaci zdrojových kódů.
    \item Popište základní problematiku tvorby Live CD.
  \end{itemize}
}

% 4. snímek
\section{Úvod do problematiky}
\frame{
  \frametitle{Úvod do problematiky}
  \setbeamercovered{transparent}
        
  OS Linux:
  % nečíslovaný seznam se zvýrazněním a automatickým zobrazením
  \begin{itemize}[<+- | alert@+>]
    \item Distribuce
    \item Linux From Scratch
  \end{itemize}

  BASH:
  \begin{itemize}[<+- | alert@+>]
    \item Jazyk BASH
    \item Integrace s OS Linux
  \end{itemize}
}

% 5. snímek
\section{Praktická část}
\frame{
  \frametitle{Ukázka rozložení s obrázkem}

%  \hyperlink{zacatek<5>}{\beamergotobutton{Návrat na úvod}}
}
\frame{
  \frametitle{Ukázka rozložení s obrázkem}

%  \hyperlink{zacatek<5>}{\beamergotobutton{Návrat na úvod}}
}

% 6. snímek
\section{Závěr}
\frame{
  \frametitle{Závěr}

}

% 7. snímek
\section{Děkuji za pozornost}
\frame{
  \frametitle{Děkuji za pozornost}
\Huge{Děkuji za pozornost\\
Otázky?}
}

\end{document}
